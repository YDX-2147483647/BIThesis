% \subsubsection{定义用户接口}
%
% \begin{macro}{\BITSetup}
% 提供用户配置的接口。
%    \begin{macrocode}
\DeclareDocumentCommand \BITSetup { m }
  { \keys_set:nn { bithesis } { #1 }}
%    \end{macrocode}
% \end{macro}
%
% \begin{macro}{\BigStar}
% 提供密级选项中需要的五角星,在普通环境中使用。
%    \begin{macrocode}
\DeclareDocumentCommand \BigStar { }
  { \ding{72} }
%    \end{macrocode}
% \end{macro}
%
% \begin{environment}{blindPeerReview}
% 用于包裹涉及个人信息的内容。
%
% 在启用盲审模式时,其中的内容会被隐藏。
%
% 本环境提供了一个可选参数,可以传入一个 bool 值,用于在盲审模式下关闭
% 隐藏行为。
%    \begin{macrocode}
  \NewDocumentEnvironment {blindPeerReview} {O{\c_true_bool} +b}
  {
    \bool_if:nTF {\g_@@_blind_mode_bool && #1} {} {
      #2
    }
  } {}
%    \end{macrocode}
% \end{environment}
%
% \begin{macro}{\cleardoublepage}
% 重定义 \tn{cleardoublepage},
% 使得偶数页面在没有内容时也不显示页眉页脚。见:
% \url{https://tex.stackexchange.com/a/1683}。
%    \begin{macrocode}
\RenewDocumentCommand \cleardoublepage { }
  {
    \clearpage
    \bool_if:NT \g_@@_twoside_bool
      {
        \int_if_odd:nF \c@page
          { \hbox:n { } \thispagestyle { empty } \newpage }
      }
  }
%    \end{macrocode}
% \end{macro}
%
% \begin{macro}{\SecretInfo}
% 用于包裹涉及个人信息的内容。
%   \begin{macrocode}
\DeclareDocumentCommand \SecretInfo { m o }
  {
    \IfValueTF {#2} {
      \@@_secret_info:nn {#1} {#2}
    } {
      \@@_secret_info:n {#1}
    }
  }
%   \end{macrocode}
% \end{macro}
%
% \begin{macro}{\MakeCover}
% 制作封面。
%    \begin{macrocode}
\DeclareDocumentCommand \MakeCover {}
  {
    \begin{blindPeerReview}[\l_@@_cover_hide_cover_in_peer_review_bool]
    \group_begin:

    % 封面使用的 thesis-type 可能与整体不同。
    \int_new:N \l_@@_thesis_type_int
    \bool_if:NTF \l_@@_cover_prefer_zh_bool {
      \int_set:Nn \l_@@_thesis_type_int {1}
    } {
      \int_set:Nn \l_@@_thesis_type_int \g_@@_thesis_type_int
    }

    \int_case:nn {\l_@@_thesis_type_int}
    {
      {1}
      {
        \currentpdfbookmark{封面}{frontmatter:cover}
        \begin{titlepage}
          \vspace*{16mm}

          \centering

          \tl_if_blank:VTF \l_bit_coverheaderimage_tl {} {
            \includegraphics[width=9.87cm]{\l_bit_coverheaderimage_tl}\\
          }

          \vspace*{-3mm}

          \zihao{-0}\textbf{\ziju{0.12}\songti{\l_@@_style_headline_tl}}\par

          \vspace{0.5em plus 1fill}

          \group_begin:
          % 中文标题
          \tl_set:Nn \l_tmpa_tl {
            \linespread{1.46}\selectfont
            \zihao{2}\textbf{\xihei:n \l_@@_value_title_tl}\par
          }
          % 英文标题
          \tl_set:Nn \l_tmpb_tl {
            \linespread{1.65}\selectfont
            \zihao{3}\textbf{\l_@@_value_title_en_tl}\par
          }

          \bool_if:NTF \l_@@_cover_reverse_titles_bool {
            \l_tmpb_tl \vspace{3mm} \l_tmpa_tl
          } {
            \l_tmpa_tl \vspace{3mm} \l_tmpb_tl
          }
          \group_end:

          \vspace{0em plus 1fill}


          \begin{spacing}{1.8}
            \begin{center}
            \tl_if_empty:NT \l_@@_cover_delimiter_tl {
              \tl_set:Nn \l_@@_cover_delimiter_tl {:}
            }
            % if not auto width, try override width
            \bool_if:NF \l_@@_cover_auto_width_bool {
              \dim_compare:nNnT {\l_@@_cover_label_max_width_dim} = {0pt} {
                \dim_set:Nn \l_@@_cover_label_max_width_dim {35mm}
              }
              \dim_compare:nNnT {\l_@@_cover_value_max_width_dim} = {0pt} {
                \dim_set:Nn \l_@@_cover_value_max_width_dim {78mm}
              }
            }

            \clist_set:Nn \l_@@_input_clist {
              {\c_@@_label_semester_tl} {\l_@@_value_semester_tl},
              {\c_@@_label_school_tl} {\l_@@_value_school_tl},
              {\g_@@_const_info_major_tl} {\l_@@_value_major_tl},
              {\c_@@_label_class_tl} {\@@_secret_info:N \l_@@_value_class_tl},
              {\c_@@_label_author_tl} {\@@_secret_info:N \l_@@_value_author_tl},
              {\c_@@_label_student_id_tl} {\@@_secret_info:N \l_@@_value_student_id_tl},
              {\c_@@_label_course_tl} {\l_@@_value_course_tl},
              {\c_@@_label_supervisor_tl} {\@@_secret_info:N \l_@@_value_supervisor_tl},
              {\c_@@_label_co_supervisor_tl} {\@@_secret_info:N \l_@@_value_external_supervisor_tl},
              {\c_@@_label_teacher_tl} {\l_@@_value_teacher_tl},
            }

            \zihao{3}

            \@@_render_cover_entry:n \l_@@_input_clist

            \end{center}
          \end{spacing}

          \vspace*{1.5em plus 1.5fill}
          \centering
          \zihao{3}\ziju{0.5}\songti{
            \tl_if_empty:NTF \l_@@_cover_date_tl {
              % 英文模板中 ctex 不会预设日期格式,但仍要保证中文封面的日期按中文习惯
              \ctexset{today=small}
              \today
            } {
              \l_@@_cover_date_tl
            }
          }
        \end{titlepage}
      }
      {2}
      {
        \begin{titlepage}
          \centering

          \tl_if_blank:VTF \l_bit_coverheaderimage_tl {} {
            \includegraphics[width=6.87cm]{\l_bit_coverheaderimage_tl}\\
          }

          \vspace{1.2mm}

          \zihao{2}\textbf{\songti{\l_@@_style_headline_tl}}

          % 外文翻译封面有两组中英文标题,行数变化范围大,因此采用 fill 比例布局
          \vspace{\stretch{1}}

          {

          \begin{spacing}{1.8}

            \tl_set:Nn \l_@@_cover_delimiter_tl {\textbf{:}}
            \bool_set_false:N \l_@@_cover_auto_width_bool
            \dim_set:Nn \l_@@_cover_label_max_width_dim {35mm}
            \dim_set:Nn \l_@@_cover_value_max_width_dim {115mm}

            \clist_set:Nn \l_@@_input_clist {
              {\songti\zihao{4}\textbf{外文原文题目}} {\l_@@_value_trans_origin_title_tl},
              {\songti\zihao{4}\textbf{中文翻译题目}} {\l_@@_value_trans_title_tl},
            }

            \zihao{-3}
            \centering

            \@@_render_cover_entry:n \l_@@_input_clist

          \end{spacing}

          }

          \vspace{\stretch{1}}

          \zihao{2}\textbf{\xihei:n \l_@@_value_title_tl}\par

          \vspace{3mm}

          \begin{spacing}{1.2}
            \zihao{3}\selectfont{\textbf{\l_@@_value_title_en_tl}}\par
          \end{spacing}

          \vspace{\stretch{0.67}}

          \begin{spacing}{1.8}
            \tl_if_empty:NT \l_@@_cover_delimiter_tl {
              \tl_set:Nn \l_@@_cover_delimiter_tl {:}
            }

            % 如果不是自动计算宽度,且用户没有自定义宽度,
            % 则尝试提供一个默认宽度。
            \bool_if:NF \l_@@_cover_auto_width_bool {
              \dim_compare:nNnT {\l_@@_cover_label_max_width_dim} = {0pt} {
                \dim_set:Nn \l_@@_cover_label_max_width_dim {35mm}
              }
              \dim_compare:nNnT {\l_@@_cover_value_max_width_dim} = {0pt} {
                \dim_set:Nn \l_@@_cover_value_max_width_dim {78mm}
              }
            }

            \zihao{3}

        % 渲染信息。
            \clist_set:Nn \l_@@_input_clist {
              {\c_@@_label_school_tl} {\l_@@_value_school_tl},
              {\c_@@_label_major_tl} {\l_@@_value_major_tl},
              {\c_@@_label_class_tl} {\@@_secret_info:N \l_@@_value_class_tl},
              {\c_@@_label_author_tl} {\@@_secret_info:N \l_@@_value_author_tl},
              {\c_@@_label_student_id_tl} {\@@_secret_info:N \l_@@_value_student_id_tl},
              {\c_@@_label_supervisor_tl} {\@@_secret_info:N \l_@@_value_supervisor_tl},
              {\c_@@_label_co_supervisor_tl} {\@@_secret_info:N \l_@@_value_external_supervisor_tl},
            }

            \@@_render_cover_entry:n \l_@@_input_clist

          \end{spacing}

          \vspace{\stretch{0.67}}
        \end{titlepage}
      }
      {3} {
        \begin{titlepage}
          \vspace*{16mm}

          \centering

          \tl_if_blank:VTF \l_bit_coverheaderimage_tl {} {
            \includegraphics[width=9.87cm]{\l_bit_coverheaderimage_tl}\\
          }

          \vspace*{-3mm}

          \zihao{1}\textbf{\ziju{0.12}\l_@@_style_headline_tl}\par

          \vspace{18mm}

          \bool_if:NT \l_@@_cover_add_titlezh_bool {
            \zihao{2}\textbf{\xihei:n \l_@@_value_title_tl}\par
            \vspace{16mm}
          }

          \zihao{2}\textbf{\xihei:n \l_@@_value_title_en_tl}\par

          \vspace{10mm}


          \begin{spacing}{1.8}
            \begin{center}
            \tl_if_empty:NT \l_@@_cover_delimiter_tl {
              \tl_set:Nn \l_@@_cover_delimiter_tl {:}
            }

            % if not auto width, try override width
            \bool_if:NF \l_@@_cover_auto_width_bool {
              \dim_compare:nNnT {\l_@@_cover_label_max_width_dim} = {0pt} {
                \dim_set:Nn \l_@@_cover_label_max_width_dim {20mm}
              }
              \dim_compare:nNnT {\l_@@_cover_value_max_width_dim} = {0pt} {
                \dim_set:Nn \l_@@_cover_value_max_width_dim {105mm}
              }
            }

            \zihao{4}

            \clist_set:Nn \l_@@_input_clist {
              {\c_@@_label_school_en_tl} {\l_@@_value_school_tl},
              {\g_@@_const_info_major_tl} {\l_@@_value_major_tl},
              {\c_@@_label_author_en_tl} {\l_@@_value_author_tl},
              {\c_@@_label_student_id_en_tl} {\l_@@_value_student_id_tl},
              {\c_@@_label_supervisor_en_tl} {\l_@@_value_supervisor_tl},
              {\c_@@_label_co_supervisor_en_tl} {\l_@@_value_external_supervisor_tl},
            }

            \@@_render_cover_entry:n \l_@@_input_clist

            \end{center}
          \end{spacing}

          \vspace*{\fill}
          \centering
          \zihao{3}\ziju{0.5}\songti{
            \tl_if_empty:NTF \l_@@_cover_date_tl {
              \today
            } {
              \l_@@_cover_date_tl
            }
          }
        \end{titlepage}
      }
      {4} {
        \make_graduate_cover:
      }
      {5} {
        \make_graduate_cover:
      }
    }
    \group_end:
    \end{blindPeerReview}
  }
%    \end{macrocode}
% \end{macro}
%
%
% \begin{macro}{\MakeOriginality}
% 原创性声明。
%    \begin{macrocode}
\NewDocumentCommand \MakeOriginality {}
  {
    \group_begin:
    \begin{blindPeerReview}[\l_@@_cover_hide_cover_in_peer_review_bool]
      \int_case:nn {\g_@@_thesis_type_int}
      {
        {1}
        {
          \currentpdfbookmark{声明}{frontmatter:originality}
          \pagestyle{BIThesis}
          \pagenumbering{gobble}

          % 原创性声明部分
          \begin{center}
            \vspace*{-2bp}
            \@@_same_page:
            \chapter*{\heiti\zihao{2}\c_@@_bachelor_label_originality_tl}
          \end{center}~\par

          % 本部分字号为小三。
          \zihao{-3}
          \c_@@_bachelor_label_originality_clause_tl

          \vspace{17mm}

          \begin{flushright}
            \c_@@_bachelor_label_originality_author_signature_tl\par
          \end{flushright}

          \vspace{16mm}

          % 使用授权声明部分
          \begin{center}
            \@@_same_page:
            \chapter*{
              \heiti\zihao{2}
              \c_@@_bachelor_label_authorization_tl
            }
          \end{center}~\par

          \c_@@_bachelor_label_authorization_clause_tl

          \vspace*{3mm}

          \begin{flushright}
            \begin{spacing}{1.65}
              \zihao{-3}
              % \hspace{5mm}\raisebox{-2ex}{\includegraphics[width=30mm]{example-image}}\hspace{5mm}
              \c_@@_bachelor_label_originality_author_signature_tl\par
              \c_@@_bachelor_label_originality_supervisor_signature_tl\par
            \end{spacing}
          \end{flushright}

          \newpage
        }
        {3} {
          \currentpdfbookmark{Statements}{frontmatter:originality}
          \pagestyle{BIThesis}
          \pagenumbering{gobble}

          % 原创性声明部分
          \begin{center}
            \vspace*{-2bp}
            \@@_same_page:
            \chapter*{
              \heiti\zihao{-2}
              \c_@@_bachelor_english_label_originality_tl
            }
          \end{center}~\par

          % 本部分字号为小四
          \zihao{-4}
          \c_@@_bachelor_english_label_originality_clause_tl

          \bigbreak

          Student~(Signature):~\dunderline[-1pt]{1pt}{\makebox[18mm]{}}~Date:\par

          \vspace{\stretch{1}}

          % 使用授权声明部分
          \begin{center}
            \@@_same_page:
            \chapter*{
              \heiti\zihao{-2}
              \c_@@_bachelor_english_label_authorization_tl
            }
          \end{center}~\par

          \c_@@_bachelor_english_label_authorization_clause_tl

          \bigbreak
          Student~(Signature):~
            \dunderline[-1pt]{1pt}{\makebox[18mm + 16bp]{}}~
            \hspace{2mm}Date:\par
          Supervisor~(Signature):~
            \dunderline[-1pt]{1pt}{\makebox[18mm]{}}~
            \hspace{2mm}Date:\par
        }
        {4} {\@@_graduate_originality:}
        {5} {\@@_graduate_originality:}
      }
    % 单独成页
    \clearpage
    \end{blindPeerReview}
    \group_end:
  }
%    \end{macrocode}
% \end{macro}
%
% \begin{macro}{\MakePaperBack}
% 生成书脊。
%    \begin{macrocode}
\NewDocumentCommand \MakePaperBack {}
  {
    % 上下各留出规定的边距,到下一页再恢复。
    % 若标题超长,自然会向上下溢出。
    %
    % 必须在顶层操作,不然影响不确定。
    % https://tex.stackexchange.com/q/718581
    %
    % 单纯`\newgeometry`再`\restoregeometry`相当于仅仅`\clearpage`,也无问题。
    \newgeometry{
      vmargin = 5cm,
    }
    \begin{blindPeerReview}[\l_@@_cover_hide_cover_in_peer_review_bool]
      \make_paper_back:
    \end{blindPeerReview}
    \restoregeometry
  }
%    \end{macrocode}
% \end{macro}
%
% \begin{macro}{\MakeTitle}
% 生成标题页。(研究生)
%    \begin{macrocode}
\NewDocumentCommand \MakeTitle {}
  {
    \begin{blindPeerReview}[\l_@@_cover_hide_cover_in_peer_review_bool]
      \@@_make_chinese_title_page:
      \@@_make_english_title_page:
    \end{blindPeerReview}
  }
%    \end{macrocode}
% \end{macro}
%
% \begin{macro}{\MakeTOC}
% 生成目录。
%    \begin{macrocode}
\DeclareDocumentCommand \MakeTOC {}
  {
    {
      \@@_if_bachelor_thesis:TF {
        \renewcommand{\baselinestretch}{1.35}
      } {
        \renewcommand{\baselinestretch}{1.56}
      }

      % 自定义目录样式
      \cs_set:Npn \contentsname {
        \fontsize{16pt}{\baselineskip}
        \l_@@_unnumchapter_style_cs:n
          \l_@@_title_font_cs:n
            {\l_@@_toc_title_tl}
        \vspace{-8pt}
      }

      \bool_if:NTF \l_@@_add_toc_to_toc_bool {
        % 添加「目录」本身到目录中,同时自动添加书签
        % 此处必须有`\phantomsection`,不然 hyperref 会把链接指向之前摘要的标题。
        \phantomsection
        \addcontentsline{toc}{chapter}{\c_@@_label_toc_en_tl}
      } {
        % 手动添加目录书签
        \currentpdfbookmark{\l_@@_toc_title_tl}{ch:toc}
      }

      % 制作目录
      \tableofcontents

      % 单独成页
      \clearpage
    }
  }
%    \end{macrocode}
% \end{macro}
%
% \begin{environment}{abstract}
% 生成摘要。
%    \begin{macrocode}
\NewDocumentEnvironment {abstract} {}
  {

    \cleardoublepage
    \linespread{1.53}\selectfont

    \@@_if_bachelor_thesis:T {
      \begin{center}
        \vspace*{-17bp}
        \heiti\zihao{-2}\textbf{
          \int_case:nn {\g_@@_thesis_type_int}
          {
            {1} {\l_@@_value_title_tl}
            {2} {\l_@@_value_trans_title_tl}
            {3} {\l_@@_value_title_tl}
          }
        }
      \end{center}

      \vspace*{2mm}
    }

    \ctexset{
      chapter/numbering = false,
    }

    \@@_if_bachelor_thesis:T {
      \ctexset{
        chapter/titleformat = {\textmd}
      }
    }

    {
      \@@_same_page:
      \bool_if:NTF \l_@@_add_abstract_to_toc_bool {
        \chapter{\c_@@_label_abstract_tl}
      } {
        \currentpdfbookmark{\c_@@_label_abstract_tl}{ch:abstract}
        \chapter*{\c_@@_label_abstract_tl}
      }
    }
    \vspace*{1mm}
    \par
  }
  {
    \par
    \vspace{4ex}
    \noindent
    \@@_if_graduate:TF {
      % 研究生模板中,“关键词”宋体小四加粗
      % 关键词为宋体小四号字。
      \textbf{\c_@@_label_keywords_tl}\l_@@_value_keywords_tl\par
    } {
      % 本科生模板中,关键词为黑体加粗
      \textbf{\heiti \c_@@_label_keywords_tl \l_@@_value_keywords_tl}\par
    }
    \newpage
  }
%    \end{macrocode}
% \end{environment}
%
% \begin{environment}{abstractEn}
% 生成英文摘要。
%    \begin{macrocode}
\NewDocumentEnvironment {abstractEn} {}
  {
    \linespread{1.53}\selectfont

    \@@_if_bachelor_thesis:T {
      \begin{spacing}{0.95}
        \centering
        \vspace*{-2bp}

        \l_@@_title_font_cs:n {
          \zihao{3}\textbf
          \l_@@_value_title_en_tl\\
        }
      \end{spacing}
      \vspace*{10mm}
    }

    \ctexset{
      chapter/numbering = false,
    }

    \@@_if_bachelor_thesis:TF {
      \ctexset{
        chapter/titleformat = {\zihao{-3}\textmd}
      }
    } {
      \ctexset {
        chapter/titleformat = {\heiti\zihao{3}\centering\textbf}
      }
    }

    {
      \@@_same_page:
      \bool_if:nTF {\l_@@_add_abstract_en_to_toc_bool} {
        \chapter{\c_@@_label_abstract_en_tl}
      } {
        \currentpdfbookmark{\c_@@_label_abstract_en_tl}{ch:abstract:en}
        \chapter*{\c_@@_label_abstract_en_tl}
      }
    }
  }
  {
    \par\vspace{3ex}\noindent
    \@@_if_graduate:TF {
      % Times New Roman小四号字,行距22磅
      % “Key Words”
      % Times New Roman小四号字加粗
      \textbf{\c_@@_label_keywords_en_tl} \l_@@_value_keywords_en_tl
    } {
      \textbf{\c_@@_label_keywords_en_tl \l_@@_value_keywords_en_tl}
    }
    \newpage
  }

%    \end{macrocode}
% \end{environment}
%
% \begin{environment}{conclusion}
% 生成结论。需要放在 \cs{macrocode} 之后。
%    \begin{macrocode}
\NewDocumentEnvironment {conclusion} {}
  {
    \ctexset{
      section/number = \arabic{section}
    }

    \@@_if_thesis_english:TF {
      \chapter{\c_@@_label_conclusion_en_tl}
    } {
      \chapter{\c_@@_label_conclusion_tl}
    }
  }
  {}
%    \end{macrocode}
% \end{environment}
%
% \begin{environment}{bibprint}
% 生成参考文献。需要放在 \cs{backmatter} 之后。
%    \begin{macrocode}
\NewDocumentEnvironment {bibprint} {}
  {
    % 设置参考文献字号为 5 号
    \renewcommand*{\bibfont}{\zihao{5}}
    % 设置参考文献各个项目之间的垂直距离为 0
    \setlength{\bibitemsep}{0ex}
    \setlength{\bibnamesep}{0ex}
    \setlength{\bibinitsep}{0ex}
    \@@_if_graduate:TF {
    } {
      % 「本科生」设置单倍行距
      \renewcommand{\baselinestretch}{1.2}
    }
    % 设置参考文献顺序标签 `[1]` 与文献内容 `作者. 文献标题...` 的间距
    \setlength{\biblabelsep}{1.7mm}

    \bool_if:NF \l_@@_style_bibliography_indent_bool {
      % 设置参考文献后文缩进为 0(与 Word 模板保持一致)
      % See: https://github.com/hushidong/biblatex-gb7714-2015
      %      如何修参考文献表的缩进?
      \cs_set:Npn \itemcmd {
        \settowidth{\lengthid}{\mkgbnumlabel{\printfield{labelnumber}}}
        %%这里是所做的调整,以下两句通过调整\lengthid来调整缩进
        \setlength{\lengthid}{0pt}
        \addtolength{\lengthid}{-\biblabelsep}
        \setlength{\lengthlw}{\textwidth}
        \addtolength{\lengthlw}{-\lengthid}
        \addvspace{\bibitemsep}%恢复\bibitemsep的作用
        \hangindent\lengthid
        \leavevmode\mkgbnumlabel{\printfield{labelnumber}}%
        \hspace{\biblabelsep}
      }
    }

    \@@_if_thesis_english:TF {
      \chapter{\c_@@_label_reference_en_tl}
    } {
      \chapter{\c_@@_label_reference_tl}
    }
  }
  {}
%    \end{macrocode}
% \end{environment}
%
% \begin{environment}{appendices}
% 生成附录。
%    \begin{macrocode}
\NewDocumentEnvironment {appendices} {}
  {
    % Used in chapter, ToC.
    \tl_new:N \l_@@_appendix_plain_label_tl
    % Used before reference label.
    \tl_new:N \l_@@_appendix_default_title_tl

    \@@_if_thesis_english:TF {
      \tl_set:Nn \l_@@_appendix_plain_label_tl {\c_@@_label_appendix_prefix_en_tl}
      \tl_set:Nn \l_@@_appendix_default_title_tl {\c_@@_label_appendix_en_tl}
    } {
      \tl_set:Nn \l_@@_appendix_plain_label_tl {\c_@@_label_appendix_prefix_tl}
      \tl_set:Nn \l_@@_appendix_default_title_tl {\c_@@_label_appendix_tl}
    }

    \bool_if:NTF \l_@@_appendices_chapter_level_bool {
      % 使用以「chapter」为顶层的附录格式

      % 仅设置 \setcounter{chapter}{0} 时,pdf 目录会索引到正文章节。
      % 因此,需要使用 \appendix 重置计数器,并将附录后面的
      % 几个章节视为特殊的附录页。
      \appendix

      \ctexset{
        chapter/numbering = true,
        chapter/name = {},
        chapter/number = \l_@@_appendix_plain_label_tl\hspace{1ex}\Alph{chapter},
        section/number = \Alph{chapter}. \arabic{section},
        subsection/number = \Alph{chapter}. \arabic{section}. \arabic{subsection},
      }

      \cs_set:Npn \thechapter {
        \Alph{chapter}
      }
    } {
      % 使用以「section」为顶层的附录格式

      % 因为不需要用到 chapter counter, 所以直接加一即可。
      \stepcounter{chapter}
      \setcounter{section}{0}
      % (与上面方法至少用一个)
      % 需要让 section 在 pdf bookmark 中输出字母而不是数字。
      % 详见 hyperref 代码。
      \gdef\theHsection{\Alph{section}}

      % 定义 \thefigure 采用节而不是章
\cs_set:Npn \thefigure {\theHsection \g_@@_label_divide_char_tl\arabic{figure}}

      \ctexset{
        section/number = \l_@@_appendix_plain_label_tl\hspace{1ex}\Alph{section},
        subsection/number = \Alph{section}. \arabic{subsection},
      }

      \cs_gset:Npn \thechapter {
        \Alph{section}
      }
      % \gdef \thechapter{\Alph{section}}

      \tl_if_blank:VTF \l_@@_appendices_title_tl {
        \chapter{\l_@@_appendix_default_title_tl}
      } {
        \chapter*{\l_@@_appendices_title_tl}
        \stepcounter{chapter}
        \tl_if_blank:VTF \l_@@_appendix_toc_title_tl {
          \addcontentsline{toc}{chapter}{\l_@@_appendix_default_title_tl}
        } {
          \addcontentsline{toc}{chapter}{\l_@@_appendix_toc_title_tl}
        }
      }
    }
  }
  {
  }
%    \end{macrocode}
% \end{environment}
%
% \begin{environment}{acknowledgements}
% 生成致谢。
%    \begin{macrocode}
\NewDocumentEnvironment {acknowledgements} {+b}
  {
    \begin{blindPeerReview}
      % 将此章节视为特殊的附录页,关闭附录编号,重定义 section 编号。
      % 不知为何,需要手动重置 section 计数器。
      \setcounter{section}{0}
      \ctexset{
        appendix/numbering = false,
        section/number = \arabic{section},
        subsection/number = \arabic{section}. \arabic{subsection},
        subsubsection/number = \arabic{section}. \arabic{subsection}. \arabic{subsubsection},
      }

      \chapter{\g_@@_const_heading_acknowledgements_tl}
      \@@_if_graduate:TF {\fangsong}{}
      #1
    \end{blindPeerReview}
  } {}
%    \end{macrocode}
% \end{environment}
%
% \begin{macro}{\Author,\AuthorEn}
% 在普通模式下,输出作者姓名。
% 在盲审模式下,输出「第 n 作者」。
%    \begin{macrocode}
\NewDocumentCommand \Author {O{1} o o}
  {
    \bool_if:NTF \g_@@_blind_mode_bool {
      % 盲审模式
      \IfValueTF {#3} {
        #3
      } {
        第\zhnumber{#1}作者
      }
    } {
      % 普通模式
      \IfValueTF {#2} {
        % 覆盖默认的 \author 命令
        #2
      } {
        % 默认采用作者姓名
        \l_@@_value_author_tl
      }
    }
  }

% 英文姓名
\NewDocumentCommand \AuthorEn {O{1} o o}
  {
    \bool_if:NTF \g_@@_blind_mode_bool {
      % 盲审模式
      \IfValueTF {#3} {
        #3
      } {
        \Ordinalstringnum{#1}~Author
      }
    } {
      % 普通模式
      \IfValueTF {#2} {
        % 覆盖默认的 \author 命令
        #2
      } {
        % 默认采用作者姓名
        \l_@@_value_author_en_tl
      }
    }
  }
%    \end{macrocode}
% \end{macro}
%
% \begin{macro}{\addpub,\addpubs}
% 添加一个或多个参考文献。
%   \begin{macrocode}
\NewDocumentCommand \addpub {m} {
  \nocite{#1}
  \addtocategory{mypub}{#1}
}

\NewDocumentCommand \addpubs {m} {
  % apply a clist
  \clist_map_function:nN {#1} \addpub
}
%   \end{macrocode}
% \end{macro}
%
% \begin{macro}{\pubsection}
% 设置小标题。
%    \begin{macrocode}
\NewDocumentCommand \pubsection {s m} {
  {
    \par
    \IfBooleanF {#1} {
      % 自增计数器
      \stepcounter{pub}
    }
    % 设置小标题,暂时没有考虑英文模式
    \noindent
    \textbf{
      \heiti{
        \IfBooleanF {#1} {\zhnumber{\thepub}、}#2
      }
    }\par
  }
}
%    \end{macrocode}
% \end{macro}
%
% \begin{environment}{publications}
% 生成攻读学位期间发表论文与研究成果清单。
%    \begin{macrocode}
\NewDocumentEnvironment {publications} {+b}
  {
    % 同时设置 omit 以及 blindPeerReview 才能跳过此章节生成。
    \begin{blindPeerReview}[\l_@@_publications_omit_bool]
      % 将此章节视为特殊的附录页,关闭附录编号,重定义 section 编号。
      % 不知为何,需要手动重置 section 计数器。
      \setcounter{section}{0}
      \ctexset{
        appendix/numbering = false,
        section/number = \arabic{section},
        subsection/number = \arabic{section}. \arabic{subsection},
        subsubsection/number = \arabic{section}. \arabic{subsection}. \arabic{subsubsection},
      }
      % 设置参考文献字号为 5 号
      \renewcommand*{\bibfont}{\zihao{5}}
      % 设置参考文献各个项目之间的垂直距离为 0
      \setlength{\bibitemsep}{0ex}
      \setlength{\bibnamesep}{0ex}
      \setlength{\bibinitsep}{0ex}
      % 设置参考文献顺序标签 `[1]` 与文献内容 `作者. 文献标题...` 的间距
      \setlength{\biblabelsep}{1.7mm}

      \bool_if:NF \l_@@_style_bibliography_indent_bool {
        % 设置参考文献后文缩进为 0(与 Word 模板保持一致)
        % See: https://github.com/hushidong/biblatex-gb7714-2015
        %      如何修参考文献表的缩进?
        \cs_set:Npn \itemcmd {
          \settowidth{\lengthid}{\mkgbnumlabel{\printfield{labelnumber}}}
          %%这里是所做的调整,以下两句通过调整\lengthid来调整缩进
          \setlength{\lengthid}{0pt}
          \addtolength{\lengthid}{-\biblabelsep}
          \setlength{\lengthlw}{\textwidth}
          \addtolength{\lengthlw}{-\lengthid}
          \addvspace{\bibitemsep}%恢复\bibitemsep的作用
          \hangindent\lengthid
          \leavevmode\mkgbnumlabel{\printfield{labelnumber}}%
          \hspace{\biblabelsep}
        }
      }

      % If in blindPeerReview mode, omit delimiters in author field.
      \bool_if:NT \g_@@_blind_mode_bool {
        % 如果有多个作者,不修改此项的话,作者与标题之间会有逗号。
        \DeclareDelimFormat[bib,biblist]{finalnamedelim}{}
        % 如果自己不是第一个作者,不修改此项的话,会在最开始有逗号。
        \DeclareDelimFormat{multinamedelim}{}
        % 如果覆盖的是英文作者,不修改此项的话,会在最开始有空格。
        \DeclareDelimFormat{bibnamedelimd}{}

        % 如果作者太多而被截断,不修改的话,会多余逗号、“等”云云。
        % 被截断的充要条件:作者数量大于通过`\BITSetup`设置的`publications/maxbibnames`。
        % 故设置标点来去掉逗号,
        \DeclareDelimFormat[bib,biblist]{andothersdelim}{}
        % 并设置本地化字符串来去掉“等”。
        \setlocalbibstring{andothers}{}
        \setlocalbibstring{andotherscn}{}
        % 另外注意,我们仍尊重`maxbibnames`和`minbibnames`,保证若开盲审时显示作者,则关盲审时也正常。
      }

      % ===== 上方定义与「参考文献」部分相同

      % 中文姓名下,此部分不参与输出。
      \cs_set:Npn \mkbibnamegiven ##1 {
        \haspartannotation{myself}{
          \bool_if:NTF \g_@@_blind_mode_bool {
            % 盲审模式,不输出内容
          } {
            % 普通模式
            \textbf{##1}
          }
        }{
          \bool_if:NTF \g_@@_blind_mode_bool {
            % 盲审模式,不输出内容
          } {
            % 普通模式
            ##1
          }
        }
      }

      \cs_set:Npn \mkbibnamefamily ##1 {
        \haspartannotation{myself}{
          % 作者为自己
          \bool_if:NTF \g_@@_blind_mode_bool {
            % 盲审模式
            \getpartannotation{myself}
          } {
            % 普通模式
            \textbf{##1}
          }
        }{
          % 作者不是自己
          \bool_if:NTF \g_@@_blind_mode_bool {
            % 盲审模式,不输出
          } {
            % 普通模式
            ##1
          }
        }
      }

      \if_cs_exist:N \c@pub {
        % 重置计数器
        \setcounter{pub}{0}
      } \else: {
        % 设置计数器
        \newcounter{pub}
      } \fi:

      % 设置参考文献的排序
      \bool_if:NTF \l_@@_publications_sorting_bool {
        % Sorting by year, name, type.
        \newrefcontext[sorting=ynt]
      } {
        % Do not sort.
        \newrefcontext
      }

      % 根据 maxbibnames 的设置,覆盖 \blx@maxbibnames 选项,保证所有作者都能显示。
      \cs_set:Npn \blx@maxbibnames {
        \l_@@_publications_maxbibnames_int
      }

      % 根据 minbibnames 的设置,覆盖 \blx@minbibnames 选项,保证所有作者都能显示。
      \cs_set:Npn \blx@minbibnames {
        \l_@@_publications_minbibnames_int
      }

      \chapter{\@@_get_const:N {publications}}
      #1
    \end{blindPeerReview}
  }
  {}
%    \end{macrocode}
% \end{environment}
%
% \begin{environment}{resume}
% 生成个人简历。
%    \begin{macrocode}
\NewDocumentEnvironment {resume} {+b}
  {
    \begin{blindPeerReview}
      % 将此章节视为特殊的附录页,关闭附录编号,重定义 section 编号。
      % 不知为何,需要手动重置 section 计数器。
      \setcounter{section}{0}
      \ctexset{
        appendix/numbering = false,
        section/number = \arabic{section},
        subsection/number = \arabic{section}. \arabic{subsection},
        subsubsection/number = \arabic{section}. \arabic{subsection}. \arabic{subsubsection},
      }
      \chapter{\@@_get_const:N{resume}}
      #1
    \end{blindPeerReview}
  }
  {
  }
%    \end{macrocode}
% \end{environment}

% \begin{environment}{symbols}
% 生成主要术语对照表。
%    \begin{macrocode}
\NewDocumentEnvironment {symbols} {}
  {
    \bool_if:NTF \l_@@_add_symbols_to_toc_bool {
      \chapter{\@@_get_const:N {symbols}}
    } {
      \chapter*{\@@_get_const:N {symbols}}
      \currentpdfbookmark{\c_@@_label_symbols_tl}{ch:symbols}
    }
    \zihao{-4}
    \begin{itemize}[
      labelwidth=2.5cm,
      labelsep=0.5cm,
      leftmargin=3cm,
      itemindent=0cm,
      % 不再在两项之间增加额外的间距(1.5 倍的行间距已经够宽了)(未来可以提供一个接口以供用户手动设置间距)
      itemsep=-0.5ex,
    ]
    \cs_set:Npn \makelabel ##1 {##1\hfil}
  }
  {
    \end{itemize}

    % 单独一页
    \clearpage
  }
%    \end{macrocode}
% \end{environment}
